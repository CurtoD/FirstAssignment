\documentclass[11pt,a4paper]{report}
\usepackage[latin1]{inputenc}
\usepackage{hyperref}
\hypersetup{
    colorlinks,
    citecolor=black,
    filecolor=black,
    linkcolor=black,
    urlcolor=black
}
\usepackage{amsmath}
\usepackage{amsfonts}
\usepackage{amssymb}
\usepackage{acronym}
\author{Diogo Curto}
\title{Low-Power Systems Evaluation}
\date{Original date: September 9, 2014 \\ Last Modified: October 8, 2014}

\begin{document}

\begin{titlepage}

\maketitle

\end{titlepage}

\tableofcontents

\chapter{Low power systems and applications}

When defining a system as being low power it comes down to the application. Each one has a definition for what it needs to be low power. In most cases, a low power application is defined by its efficiency.  
The system needs to get the most work done in the less time as possible, with the minimum use of CPU while working as slow as possible and consume only the strictly necessary power.\cite{Wong2009}


With this thoughts in mind, is clear that low power systems require optimization at all levels, from the microcontroller architecture up through the application layer.

Throughout this chapter will be introduced the main power categories and design techniques for the microcontroller based systems and applications to be as efficient as possible.

\section{Power Categories}

There are two real factors involved in the application design: dynamic power consumption when the application is running and static power for when it is asleep.\cite{SedraSmith}

Dynamic power is affected mainly by the supply voltage and the charging and discharging of capacitances at the clock frequency. This is the power consumed when the CPU is running and processing data. In a more general view, dynamic power is not only associated to the CPU itself, but also to the peripherals, both internal and external. This is important since most microcontrollers have the ability to shutdown internal peripherals completely in order to save power. External peripherals can also be shutdown by using, for instance, I/O pins to power those devices, or even to control a transistor working as switch.


Static power is caused by transistor leakage at the chip level and is highly dependable on voltage and temperature. This is considered when the CPU is in standby (idle) and normally waiting for an event to occur. In battery applications this is the state in which the CPU stays longer and therefore consumes the most significant battery power. Most applications need memory to work properly and therefore data retention power must be taken into account when discussing static power.
 
In terms of embedded systems, the most common practice is to discuss current consumption of each part and therefore, over this text, device characterization will be expressed in current consumption. 

\section{Design Techniques}

When designing a low power system the first step is normally to do a power budget. Is necessary to evaluate
the needed or allowable power modes and estimate for how long the system will be running to accomplish the determined objectives. It is also important to consider the use of the peripherals: the time they will run in each different power mode and so on.

After this is necessary to calculate the current consumption of each mode, the time spent on each mode and get an overall calculation of peak and average power for the application.

This allows to get an overall view in what is needed from the power supply and the expected battery life. It also allows to get a good view of what parts of the system need more focus in order to minimize current consumption.


\bibliography{TESE}
\bibliographystyle{ieeetran}

\end{document}

